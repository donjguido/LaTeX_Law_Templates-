% lettre.tex

\noindent \LieuAuteur, \today\\[2em]
\noindent \textbf{SOUS TOUTES RÉSERVES}\\[2em]
\noindent \ModeEnvoi\\

\noindent \NomDestinataire, \TitreDestinataire\\
\noindent \EntrepriseDestinataire\\
\noindent \AdresseDestinataireUne\\
\noindent \AdresseDestinataireDeux\\

\noindent \textbf{Objet :\quad Mise en demeure}\\

Madame,\\
Monsieur,

La présente est pour vous informer que je vous réclame la somme de \textbf{\MontantReclame} pour les raisons suivantes :

\begin{itemize}
    \item \DescriptionProbleme
\end{itemize}

Je vous mets donc en demeure de me payer la somme de {\MontantReclame} \textbf{dans un délai de 10 jours}.

Je vous mets aussi en demeure de me remettre tel objet, qui m’appartient, et que vous avez en votre possession, dans un délai de 10 jours.

Dans le cas contraire, des procédures judiciaires pourront être intentées contre vous sans autre avis ni délai.

Je vous informe que j’examinerai toute proposition de recourir à la médiation ou à la négociation avant de m’adresser au tribunal.

Veuillez agir en conséquence.

\vspace{3em}

\NomAuteur\\
\AdresseAuteurUne, \AdresseAuteurDeux\\
\TelephoneAuteur
